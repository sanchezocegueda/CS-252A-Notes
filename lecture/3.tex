
\setcounter{section}{0}
\counterwithout{section}{chapter}

\textbf{January 28$^{\text{th}}$}

\section{RISC-V}

Recall the different phases of RISC-V instruction execution

\begin{itemize}
    \item Instruction Fetch
    \item Instruction Decode
    \item Register Fetch
    \item ALU Operations
    \item \textit{Optional} Memory Operations
    \item \textit{Optional} Register Writeback
    \item Calculate Next Instruction Address
\end{itemize}

\section{Microcode Sketches}

Instruction fetch:

\begin{enumerate}
    \item $M_A, A := PC$
    \item $PC := A + 4$
    \item wait for memory
    \item $IR := Mem$
    \item dispatch on opcode
\end{enumerate}

Note that we can latch the PC into the operand of the ALU because we can already have it.
The other reason why we do $PC := A+4$ as the second instruction, is that we have an opportunity to hide the number of cycles that it takes to do $PC := A +4$ by doing it in the shadow of the memory operation.

\section{Pure ROM Implementation}
This is a very redundant and thus very inefficient design.
However, it is nice because it is easy to 

Address bits:

$$|\mu_{address}| = |\mu_{PC}| + |opcode| + 1 + 1$$

Data bits:

$$|data| = |\mu_{PC}| + |control\;signals| = |\mu_{PC}| + 18$$

Total ROM size = 

$$2^{|address|} \times |data|$$

No logic, no RAM, huge on ROM.

\subsection{Reducing Control Store Size}
There are two main approaches towards optimizing this ROM.
The first is to shrink is height, and the second is to shrink its width.
To shrink its height, we want to make the 
To reduce its width, we want to get less control signals.


\section{Single-Bus RISC-V Microcode Engine}

The idea is to add some logic to multiplex some of the $\mu_{PC}$ values.
This reduces the number of bits that we need to encode into the address.

Moral of the story: 
A little bit of logic will allow you to reduce your ROM size quite a bit.

This is a bit-addressable ROM, so the atomic increment of PC is 4.

All of this is microarchitectural, it is not adhering to any ISA-level rules.

\section{Nanocoding}
Nanocoding is a way to coding that tries to implement the best parts of both horizontal and vertical microcode.
It adds a level of indirection as subroutines for microcode.
It adds an indirection table.

\subsection{Horizontal Microcode}


\subsection{Vertical Microcode}



One way to fix a bug in an implementation is by patching the microcode ROM.

\section{Pipelining}


Iron law of performance

$$\frac{Time}{Program} = \frac{Instructions}{Program} \cdot \frac{Cycles}{Instruction} \cdot \frac{Time}{Cycle}$$

the instructions per program depend on source code, compiler, and ISA

CPI depends on ISA and $\mu$architecture

time per cycle depends upon the $\mu$architecture and base technology (physics)

